\documentclass{article}

\usepackage{amsmath,amssymb,amsthm}
\usepackage{stmaryrd}
\usepackage{supertabular}
\usepackage{fullpage}
\usepackage{xcolor}
\usepackage{listings}
\usepackage{fancyvrb}
\usepackage{latexsym}
\usepackage{bm}

\input{temp/defns}

\renewcommand{\ottkw}[1]{\mathsf{#1}}

\DeclareMathAlphabet{\mathitbf}{OT1}{cmr}{bx}{it}

\definecolor{dkblue}{rgb}{0,0.1,0.5}
\definecolor{dkgreen}{rgb}{0,0.3,0}
\definecolor{dkred}{rgb}{0.6,0,0}

\newcommand{\comm}[3]{\textcolor{#1}{[#2: #3]}}
\newcommand{\ch}[1]{\comm{violet}{CH}{#1}} % Catalin
\newcommand{\bcp}[1]{\comm{dkred}{BCP}{#1}} % Benjamin
\newcommand{\dd}[1]{\comm{dkblue}{DD}{#1}} % Delphine


\begin{document}

\ch{See notes.org for the text, this is just the rules}

\section{Syntax}

\ottmetavars\\[0pt]

\ottgrammartabular{
\ottframe\ottinterrule
\ottp\ottinterrule
\ottn\ottinterrule
\ottu\ottinterrule
\ottinstr\ottinterrule
\otte\ottinterrule
%\ottopcode\ottinterrule
\ottS\ottinterrule
%\ottformula\ottinterrule
%\ottterminals\ottinterrule
%\ottAbstractHighLevelSmallStep\ottinterrule
%\ottjudgement\ottinterrule
%\ottuserXXsyntax\ottafterlastrule
}\\[5.0mm]

\section{High-level abstract machine}

\ottdefnsAbstractHighLevelSmallStep

\section{Slightly less high-level abstract machine}

This is similar to the quasi-abstract machine in notes.org, just that
the rules and system calls are not extracted out of the semantics. The
tagging mechanism is, however, made explicit as in the quasi-abstract
machine.
%
\ch{At this point the goal of this machine is solely explorative and
  didactic. I have no clue if something like would help better
  structure some proof.}

\ottgrammartabular{
\ottT\ottinterrule
\ottTy\ottinterrule
\ottaa\ottinterrule
\ottaS\ottinterrule
}\\[5.0mm]

\ottdefnsAbstractSlightlyLessHighLevelSmallStep

\end{document}

