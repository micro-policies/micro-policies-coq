\documentclass[12pt]{amsart}

\newif\ifdraft\drafttrue

\newcommand{\asz}[1]{\ifdraft\textcolor{violet}{[ASZ: #1]}\fi}

\usepackage[utf8]{inputenc}
\usepackage[T1]{fontenc}
\usepackage{lmodern}
\usepackage{microtype}
\usepackage{fullpage}
\usepackage{xcolor}

\usepackage{amsmath,amsfonts,amsthm}
\usepackage{thmtools}
\usepackage{mathtools}

\declaretheoremstyle[
  headfont=\normalfont\bfseries,
  notefont=\normalfont\bfseries, notebraces={(}{)},
  bodyfont=\normalfont%\itshape
]{myplain}

\declaretheorem[style=myplain]{theorem}
\declaretheorem[style=myplain]{lemma}
\declaretheorem[style=myplain]{definition}

% Cleveref (must be loaded late---after hyperref, no less!)
\usepackage[noabbrev,capitalize,nameinlink]{cleveref}
\newcommand{\creflastconjunction}{, and~}

\newcommand*{\latin}[1]{\emph{#1}}
\newcommand*{\term}[1]{\emph{#1}}
\newcommand*{\N}{\mathbb{N}}
\newcommand*{\Z}{\mathbb{Z}}

% Colors
\definecolor{dkblue}{rgb}{0,0.1,0.5}
\definecolor{dkgreen}{rgb}{0,0.6,0}
\definecolor{dkred}{rgb}{0.6,0,0}
\definecolor{dkpurple}{rgb}{0.7,0,0.4}
\definecolor{olive}{rgb}{0.5, 0.5, 0.0}
\definecolor{teal}{rgb}{0.0,0.5,0.5}

\newcommand{\comm}[3]{\ifdraft\textcolor{#1}{[#2: #3]}\fi}
\newcommand{\bcp}[1]{\comm{dkpurple}{BCP}{#1}} % Benjamin
\newcommand{\ch}[1]{\comm{dkgreen}{CH}{#1}} % Catalin
\newcommand{\apt}[1]{\comm{dkred}{APT}{#1}} % Andrew
\newcommand{\aaa}[1]{\comm{blue}{AAA}{#1}} % Arthur

\ifdefined\SFIMacros\else
\def\SFIMacros{\SFIMacros}

% Variables.  VERY LOOSELY, a capital letter means it's ``composite'' in a
% certain sense.
\newcommand*{\Machine}{\mathcal{M}}
 
\newcommand*{\pc}{\mathit{pc}}
\newcommand*{\Registers}{R}
\newcommand*{\Memory}{M}
\newcommand*{\Compartments}{C}
 
\newcommand*{\AddressSpace}{A}
\newcommand*{\JumpTargets}{J}
\newcommand*{\SharedMemory}{S}
 
\newcommand*{\compartment}{c}
 
\newcommand*{\pointer}{p}
 
\newcommand*{\stepsto}{\to}

\newcommand*{\opcode}[1]{\mathsc{#1}}
\newcommand*{\operator}[1]{\mathsc{#1}}

% \CFoo = Concrete Foo
 
\newcommand*{\concrete}[1]{\mathtt{#1}}
 
\newcommand*{\CMachine}{\mathcal{M}_{\concrete{C}}}
 
\newcommand*{\Cpc}{\concrete{pc}}
\newcommand*{\CRegisters}{\concrete{R}}
\newcommand*{\CMemory}{\concrete{M}}
\newcommand*{\CCompartments}{\concrete{C}}
 
\newcommand*{\Ccompartment}{\concrete{c}}
 
\newcommand*{\CIncoming}{\concrete{I}}
\newcommand*{\CWriters}{\concrete{W}}

\newcommand*{\Label}{\concrete{L}}
 
\newcommand*{\Cpcstate}{\concrete{S}}
 
\newcommand*{\Cpointer}{\concrete{p}}

\newcommand*{\labeled}{\mathbin{\text{\small @}}}

\newcommand*{\INTERNAL}{\concrete{INTERNAL}}
\newcommand*{\JUMPED}{\concrete{JUMPED}}
\newcommand*{\REG}{\concrete{REG}}

\newcommand*{\canexec}{⊢_{\!\!\!\stepsto}}

\fi % \SFIMacros

\input{sfi-rules}

\begin{document}

\section{An outline of the software fault isolation machine}

The goal of the software fault isolation (SFI) machine is compartmentalization:
allowing a machine to subdivide its address space into multiple
mutually-untrusting compartments.  This division is hierarchical: each
compartment can, during execution, split off a sub-compartment out of its
address space.

To get this, we first define a general notion of what a compartment is.

\begin{definition}[Compartment]\label{def:compartment}
  A \term{compartment} $\compartment$ is a triple
  $(\AddressSpace,\JumpTargets,\SharedMemory)$ of:
  \begin{enumerate}
  \item An \term{address space} $\AddressSpace \subseteq \N$ of addresses that
    it is allowed to execute, read from, and write to;
  \item A set of \term{jump targets} $\JumpTargets \subseteq \N$ of addresses
    (presumably not in its address space) that it is allowed to jump to; and
  \item A set of \term{shared memory locations} $\SharedMemory \subseteq \N$ of
    addresses (presumably not in its address space) that it is allowed to write
    to.
  \end{enumerate}
  All compartments are permitted to read all memory locations.
\ch{This: (a) is super dangerous (see Heartbleed);
%
  (b) is kind of silly; we have a mechanism that can do it with no
  additional overhead, still you chose to allow horrible
  vulnerabilities for ``backwards'' compatibility with a 20+ years old
  paper;
%
  (c) seems to contradict point (1).}\asz{These are fair points.  But this is a
  really easy aspect of the design to change, so I didn't put much thought into
  it after making the choice to stick with the Wahbe~\latin{et al.}\ model.
  It's not that I'm opposed to changing it!}\ch{Let's change it then, and claim
  is as a contribution}
  We denote a set of compartments by
$\Compartments$.
\end{definition}

The abstract machine model for SFI is given by a simple register-based machine
with a single memory holding both instructions (encoded as numbers) and data.
It executes the same set of instructions that we have defined above\asz{since
  this will get plugged into the real paper rather than living independently as
  it does now}, which have been encoded into their concrete numeric
representations, in addition to one specialized instruction $\opcode{isolate}$
for creating new compartments.\ch{Nitpicking: the way we abstract
  monitor operations / system calls is not to new instructions.}\asz{Yes, I've
  been meaning to change this but haven't gotten around to it.}

\ch{The usage of $\N$ here is a bit disturbing. Our words are not
  nats, but bit strings.}%
\asz{True, this was an oversight.  Nothing here actually relies on $\N$, though,
  and the Coq is in terms of \texttt{word}s.}

\begin{definition}[SFI machine state]\label{def:sfi-machine-state}
  An \term{SFI machine state} $\Machine$ is a quadruple
  $(\pc,\Registers,\Memory,\Compartments)$ of
  \begin{enumerate}
  \item A \term{program counter} $\pc \in \N$ pointing at the to-be-executed
    instruction;
  \item A finite set of \term{registers} $\Registers : \mathit{reg} \to \N$;%
    \asz{not sure exactly how to formalize the domain of $\Registers$ here}
  \item A \term{memory} $\Memory : \N \to \N$ containing both code and data; and
  \item A set of \term{compartments} $\Compartments$.
  \end{enumerate}
\end{definition}

To enforce compartmentalization, we often need to check that an address
$\pointer$ lies in a particular compartment's address space; we thus define the
judgement $\Compartments \vdash \pointer \in (\AddressSpace, \JumpTargets,
\SharedMemory)$ which is true exactly when $\pointer$ lies in the address space
$\AddressSpace$ of some compartment in $\Compartments$, as we can see in
\cref{fig:in-compartment}.  This figure also defines a similar judgement
$\Compartments \vdash \pointer_1,\cdots,\pointer_k \in (\AddressSpace,
\JumpTargets, \SharedMemory)$ to check that multiple addresses all lie in the
same compartment.
%
\ch{This strange judgement seems to be mixing 2 different jugdments,
  one that {\bf nondeterministically} looks up a component
  $(\AddressSpace, \JumpTargets, \SharedMemory) \in \Compartments$,
  and another one that tests whether an address is within bounds $p
  \in \AddressSpace$.}%
\asz{The reason it's one judgement for both is because you can't just ``look up
  a component'' -- you have to look it up by some sort of key.  Here, the key is
  the pointer $\pointer$, since that's what we need in practice.  (I'm open to
  notational suggestions.)}
%
\ch{What's wrong with $(\AddressSpace, \JumpTargets, \SharedMemory)
  \in \Compartments$ and $p \in \AddressSpace$? I think
  nondeterminism does the trick perfectly, without introducing
  real nondeterminism in the semantics when the components are disjoint.}

\begin{figure}
  \ottdefninXXcompartment{}
  \par\vspace{\baselineskip}
  \ottdefnmanyXXinXXcompartment{}
  \vspace{-\baselineskip}
  \caption{The definition of what it means for addresses to lie inside
    compartments.}
  \label{fig:in-compartment}
\end{figure}

Given this judgement, most of the step relations are immediately comprehensible;
those instructions that involve neither memory accesses nor direct jumps behave
the same as before, except with the additional check that $\Compartments \vdash
\pc,\pc+1 \in (\AddressSpace, \JumpTargets, \SharedMemory)$; these simple rules
are presented in \cref{fig:sfi-step-simple}.  The one exception is the
\ottdrulename{step\_bnz\_nonzero} rule for relative branches; to jump a distance
of $x$ from the current $\pc$, the check is of course $\Compartments \vdash
\pc,\pc+x \in (\AddressSpace, \JumpTargets, \SharedMemory)$ instead.  These
semantics\ch{rules?}
guarantee that as long as execution begins in one compartment, it is
not allowed to leave it except through certain permitted jumps.

\begin{figure}
  \begin{ottdefnblock}{$\mathcal{M} \ottsym{→} \mathcal{M}'$}
                      {\ottcom{$\mathcal{M}$ steps to $\mathcal{M}'$}}
    \ottusedrule{\ottdrulestepXXnop{}}\medskip
    \ottusedrule{\ottdrulestepXXconst{}}\medskip
    \ottusedrule{\ottdrulestepXXmov{}}\medskip
    \ottusedrule{\ottdrulestepXXbinop{}}\medskip
    \ottusedrule{\ottdrulestepXXbnzXXzero{}}\medskip
    \ottusedrule{\ottdrulestepXXbnzXXnonzero{}}
  \end{ottdefnblock}\vspace{-\baselineskip}
  \caption{The standard portion of the SFI machine semantics.}
  \label{fig:sfi-step-simple}
\end{figure}

The $\opcode{jump}$ instruction on the SFI machine is where the semantics of the
compartments are the clearest, as we can see in \cref{fig:sfi-step-jump}.  We
require that the target of a jump either lie in the same compartment
($\Registers(r_{\mathit{tgt}}) \in \AddressSpace$) or be one of the approved
entry points of another compartment that this compartment has been given access
to ($\Registers(r_{\mathit{tgt}}) \in \JumpTargets$).  It is this second case
alone that allows for execution to transfer between compartments.  On this
machine, if we are using this to simulate a function call,\ch{That's a
  JAL instruction not a JUMP}\asz{Either way -- the gains from JAL aren't huge
  here because you can't automatically jump back.}\ch{This is a limitation of
  your policy, while JAL is part of most calling conventions on RISC
  machines.  We should stick to that, since otherwise the callee has no way
  of knowing to which place it has to return. How to make the policy to allow
  such returns is another issue, for which I think we need a better answer,
  but still a different issue.}
the target
compartment must have the originating compartment as a permissible jump target
\asz{a difference from the original paper, because we want to avoid protected
  stacks}\ch{Protected stack would be a nice alternative if we can do
  it with tags};
registers can be used to pass function arguments and results, and are
all caller-save.  It is necessary to clear all registers before
inter-compartment jumps if the program wants to avoid leaking data between
compartments.\ch{This seems like a very inefficient proposition, and
  we're not enforcing this calling convention in any way.}\asz{I agree, but I
  couldn't come up with a better option.  It's not unsafe \latin{per se},
  because we still can't jump/write to bad locations, but leaking information
  isn't great.  But without an internalized notion of ``private information'', I
  don't see how to do better.}

\begin{figure}
  \begin{ottdefnblock}{$\mathcal{M} \ottsym{→} \mathcal{M}'$}
                      {\ottcom{$\mathcal{M}$ steps to $\mathcal{M}'$}}
    \ottusedrule{\ottdrulestepXXjump{}}
  \end{ottdefnblock}\vspace{-\baselineskip}
  \caption{Jumping on the SFI machine.}
  \label{fig:sfi-step-jump}
\end{figure}

When accessing memory, we must also be compartment-aware.  For this model of
SFI, loads and stores are \emph{asymmetric}, as we can see in
\cref{fig:sfi-step-memory}.  The $\opcode{load}$ instruction standard in that it
does \emph{not} check that the memory location it's reading is within the same
compartment as the executing instruction; this means all memory on the machine
is world-readable (which is particularly useful for passing large function
arguments\ch{ and for Heartbleed}).
The $\opcode{store}$ instruction, on the other hand, has a check
similar to that of $\opcode{jump}$: for the most part, compartments are only
allowed to write to their own address space ($\Registers(r_{\mathit{pdest}}) \in
\AddressSpace$); however, if they've been given access to an approved shared
variable, then they can write to it as well ($\Registers(r_{\mathit{pdest}}) \in
\SharedMemory$).

\begin{figure}
  \begin{ottdefnblock}{$\mathcal{M} \ottsym{→} \mathcal{M}'$}
                      {\ottcom{$\mathcal{M}$ steps to $\mathcal{M}'$}}
    \ottusedrule{\ottdrulestepXXload{}}\medskip
    \ottusedrule{\ottdrulestepXXstore{}}
  \end{ottdefnblock}\vspace{-\baselineskip}
  \caption{Memory access on the SFI machine.}
  \label{fig:sfi-step-memory}
\end{figure}

The final feature this machine requires is the ability to create \emph{new}
compartments, which is done via the slightly hideous-looking instruction
$\opcode{isolate}$, as seen in \cref{fig:sfi-step-isolate}.  The
$\opcode{isolate}$ instruction allows a compartment to carve a portion of itself
off and use that as a brand-new compartment.  The instruction's effects break
down as follows:
\begin{enumerate}
\item The address space of the new compartment is a contiguous subset of the
  address space of the parent; the start- and endpoints of the new address space
  are passed in via $r_{\mathit{src1}}$ and $r_{\mathit{src2}}$, respectively,
  and the former must be smaller than the latter.

\item The valid jump targets for the new compartment are some set of addresses
  that are either (a) in the parent's address space or (b) valid jump targets
  for the parent.  A pointer to them is passed in the register
  $r_{\mathit{psrc3}}$.  The new jump targets must have been assembled in memory
  in a standard length-encoded vector format: the initial value is the number of
  jump targets, which is to be followed by that many addresses.

\item The shared memory locations available to the new compartment are some set
  of addresses that are either (a) in the parent's address space or (b) shared
  memory locations available to the parent.  A pointer to them is passed in
  the register $r_{\mathit{psrc4}}$.  The new shared memory locations must have been
  assembled in memory in the same length-encoded vector format used for the jump
  targets.

\item The parent compartment is modified: the addresses used for the child
  compartment are taken out of the parent's address space, and the initial
  address of the child is added as a valid jump target.
  \ch{Some strange asymmetry I don't like: the child has more access to the
    parent than the parent has to the child.}%
  \asz{I must have explained something badly here, or written something down
    wrong -- that's not true (or at least, it shouldn't be).  What are you
    referring to?}
  \ch{As far at I understand isolate allows the parent to choose to
    which of its addresses to give access to the child, but it doesn't
    allow the parent to choose to which child addresses to give itself
    access. This is a strange asymmetry, given that the parent is the
    one creating the new component and still gets strictly less access
    to it than the new component can get over the parent.}

\item Finally, the new compartment is also inserted into the machine's set of
  permitted compartments.
\end{enumerate}
This instruction is the only way that either the set of compartments or an
individual compartment can be modified.

\begin{figure}
  \begin{ottdefnblock}{$\mathcal{M} \ottsym{→} \mathcal{M}'$}
                      {\ottcom{$\mathcal{M}$ steps to $\mathcal{M}'$}}
    \ottusedrule{\ottdrulestepXXisolate{}}
  \end{ottdefnblock}\vspace{-\baselineskip}
  \caption{Creating a new compartment on the SFI machine.}
  \label{fig:sfi-step-isolate}
\end{figure}

\section{Definitions we may use and theorems we may prove}

In order to guarantee that compartmentalization is meaningful, we will want to
guarantee that all compartments are in control of distinct portions of the
machine state, excepting those jump targets and memory locations they have
voluntarily shared.
%
\ch{Children don't {\em voluntarily} share things (their start
  address) to their parents, and that's just fine.}

\begin{definition}[Non-overlapping compartments]\label{def:non-overlapping}
  A set $\Compartments$ of compartments is \term{non-overlapping} when, for all
  pairs of compartments $(\AddressSpace, \JumpTargets, \SharedMemory),
  (\AddressSpace', \JumpTargets', \SharedMemory') \in \Compartments$, the two
  address spaces do not coincide: $\AddressSpace \cap \AddressSpace' =
  \emptyset$.
\end{definition}

We can guarantee that this notion is preserved by execution.

\begin{lemma}[Preservation of non-overlapping compartments]
\label{lem:non-overlapping}
  Given two SFI machine states $\Machine$ and $\Machine'$ such that $\Machine
  \stepsto \Machine'$, if $\Machine$'s compartments are non-overlapping, then so
  are $\Machine'$'s.
\end{lemma}

Since machines are hierarchical, we can define terms expressing how they are
related.

\begin{definition}[Inter-compartment relations]
\label{def:inter-compartment-relations}
  Given two compartments $\compartment$ and $\compartment'$, we have the direct
  relationships arising from $\opcode{isolate}$:
  \begin{enumerate}
  \item $\compartment$ is the \term{direct parent} of $\compartment'$, or
    $\compartment'$ is the \term{direct child} of $\compartment$, if
    $\compartment$ executed an $\opcode{isolate}$ instruction that created
    $\compartment'$.
    
  \item $\compartment'$ is a \term{direct descendant} of $\compartment$ if
    $\compartment$ executed an $\opcode{isolate}$ instruction that created a
    \emph{different} compartment and updated $\compartment$ to be
    $\compartment'$; this symmetrically makes $\compartment$ a \term{direct
    ancestor} of $\compartment'$.
  \end{enumerate}

  We then define the \term{(indirect) descendant/ancestor} relation to be the
  reflexive transitive closure of the direct descendant/ancestor relation, and
  the \term{(indirect) parent/child} relations to be the direct parent/child
  relationships except with the requirement that $\compartment$ created an
  ancestor of $\compartment'$.

  We define the \term{cone} of a compartment $\compartment$ to be those
  compartments that are reflexively or transitively a child or descendant of
  $\compartment$.
\end{definition}

We can then guarantee that compartments are only visible to their cone.

\begin{lemma}[In-cone changes\asz{needs a better name}]\label{lem:in-cone-changes}
  Suppose we have a machine state $\Machine$ with a compartment $\compartment =
  (\AddressSpace, \JumpTargets, \SharedMemory)$ with an address $\pointer \in
  \AddressSpace$ that is not a valid shared memory location for any of its
  ancestors.  Then during any execution $\Machine \stepsto^* \Machine'$, the
  only modifications to $\pointer$ can come from $\compartment$ or any
  compartment in its cone.  Similarly, if $\pointer$ is not a valid jump target
  for any ancestor of $\compartment$, then execution steps that set the $\pc$ to
  $\pointer$ must have come from $\compartment$ or any compartment in its cone.%
  \asz{Probably needs to be stated more cleanly, too.}
\end{lemma}

We have certain constraints on the steps an SFI machine may take: either they
remain within the compartment, or they have been explicitly blessed.

\begin{lemma}[Permitted execution steps]\label{lem:permitted-steps}
  Given two SFI machine states $\Machine$ and $\Machine'$ such that $\Machine
  \stepsto \Machine'$ and $\Compartments \vdash \pc \in \compartment =
  (\AddressSpace, \JumpTargets, \SharedMemory)$.  Then either $\Compartments'
  \vdash \pc' \in \compartment$ or $\pc' \in \JumpTargets$.  \asz{Modulo a
    slight detail about what happens if the executed instruction was an
    $\opcode{isolate}$ that remove $\pc+1$ from the instruction set -- but that
    means that stepping further should crash, so it's ok!  The trickiest detail
    about this is probably figuring out exactly how to phrase this lemma so that
    it's most useful.}
\end{lemma}

We may state the following theorems about our SFI machine that characterize its
desired behavior.

% Restate to just talk about the initial set A, then we get the
% compartmentalized version as a corollary.
\begin{theorem}[Mutual independence]\label{thm:mutual-independence}
  Between executions of a compartment, its strictly local state does not change.
  More formally: suppose that we have an execution trace of an SFI machine given
  by \[ (\pc,\Registers,\Memory,\Compartments) = \Machine \stepsto \Machine''_1
  \stepsto^* \Machine''_n \stepsto \Machine' =
  (\pc',\Registers',\Memory',\Compartments') \] such that both $\pc$ and $\pc'$
  lie inside the same compartment $\compartment =
  (\AddressSpace,\JumpTargets,\SharedMemory)$, but none of the $\pc''_i$ do.
  Then \[ \Memory[\AddressSpace \setminus \SharedMemory] =
  \Memory'[\AddressSpace \setminus \SharedMemory]. \]%
  \asz{This needs to be sharpened to deal with the case where $\Machine$ splits
    off a new compartment in its last step.  On the other hand, that can't put
    execution into a new compartment in the current model, so it may just barely
    be safe.  Proving the theorem will reveal all :-)}
\end{theorem}

\begin{theorem}[Compartment entry points]\label{thm:entry-points}
  We can only enter a compartment by some address it has willingly exposed as a
  jump target.  More formally: suppose that we have an execution trace of an SFI
  machine given by \[ \Machine_0 \stepsto^* \Machine \stepsto \Machine' \] where
  $\Compartments \vdash \pc \in \compartment = (\AddressSpace, \JumpTargets,
  \SharedMemory)$.  \asz{We also need some sort of ``initial state'' guarantee.}
  Then either $\pc' \in \AddressSpace$ as well, or $\Compartments' \vdash \pc'
  \in \compartment' = (\AddressSpace', \JumpTargets', \SharedMemory')$ at some
  prior step, the machine executed $\opcode{isolate}$ from compartment
  $\compartment'$ with $\pc'$ in the jump targets vector.
\end{theorem}

\ch{I'm quite surprised that none of these properties reseambles
  non-interference. You do seem to protect the integrity (but for
  silly reasons not confidentiality) of components from other
  components, up to certain well-defined ``declassification''
  steps/points [the register contents are endorsed on calls].}%
\asz{I think it's because SFI is about labeling addresses but not values.
\ch{I still don't see why this means we can't achieve a non-interference
  like property with this more coarse-grained mechanism.}
  Mutual independence does have some of the NI feel to me, though.}
\ch{Mutual independence still talks about single runs, while
  non-interference talks about pairs of runs. Mutual independence
  might be a nice lemma in a non-interference proof :)}

\section{Differences from the original  paper}

While this machine tries to hew closely to ``Efficient Software-Based Fault
Isolation''~(Wahbe et.\ al,~1993), there are some differences.
\begin{enumerate}
\item Wahbe et.\ al instrument existing programs; this machine is designed for
  programs working with SFI from the ground up.

\item Wahbe et.\ al ultimately use a model which causes disallowed
  inter-compartment jumps and stores to have essentially random
  intra-compartment effects, in the name of efficiency; we stick with enforcing
  failure on compartmentalization violations.

\item Wahbe et. al are modeling a system where there is one trusted program
  which calls out to mutually-unaware untrusted compartments; we model a system
  with various mutually-untrusted compartments which can create children that
  may or may not be mutually aware.
  
\item Wahbe et.\ al support inter-compartment procedure calls by marshalling
  data analogously to an ordinary RPC, complete with trusted calling and
  returning stubs; we do not have trusted code and do not wish to support a
  protected stack, so we only support inter-compartment jumps and require the
  user to manage simulating RPCs themself.

\item Wahbe et.\ al supports detecting RPC crashes and can turn them into
  graceful continuation of the calling compartment; we have neither RPCs nor the
  ability to detect program crashes.
\end{enumerate}

The first two of these differences arise naturally from the nature of our
tag-based architectural design as opposed to Wahbe et.\ al's instrumentation of
existing code on existing architectures.  The third arises similarly, and
generalizes the existing behavior.  The fourth and fifth differences are sadly
unavoidable given some of our other design goals.\asz{This last needs to be
  rephrased.}

\section{Label model}

Reifying this SFI model to the concrete machine is essentially done by turning
the definition of a compartment ``inside out'' -- the abstract machine has a
global state that keep track of a set of compartments and, for each one, the
other locations it's allowed to jump or write to; the concrete label model has
sprinkles this state over the entire machine memory, tagging each datum with
information about which compartment it belongs to and which other compartments
are allowed to jump or write to it.  Every location in memory is tagged with an
SFI data label.\asz{Could be more clearly explained.}

\begin{definition}[SFI data label]\label{def:sfi-data-label}
  An \term{SFI data label} is a triple $(\Ccompartment,\CIncoming,\CWriters)$ of
  \begin{enumerate}
  \item A compartment ID $\Ccompartment \in \Z_{2^{32}}$ to which the tagged
    datum belongs.
  \item A set of \term{incoming compartments} $\CIncoming \subseteq
    \Z_{2^{32}}$, which is the set of other compartments that are allowed to
    jump to this location.
  \item A set of \term{writers} $\CWriters \subseteq \Z_{2^{32}}$, which is the
    set of other compartments that are allowed to write to this location.
  \end{enumerate}
\end{definition}

The concrete program counter $\Cpc$ is always tagged with a single compartment
ID that reflects the compartment that just executed an instruction; however, in
order to permit inter-compartment jumps, we must record whether that instruction
was a $\opcode{jump}$.

\ch{It seems that this way of protecting jumps will have the same
  delayed failure behavior as for CFI (actually mentioned bellow). The
  lessons learned there could probably help here too: e.g.\ there we
  add an extra step to the abstract machine and prove a standard
  simulation property.}%
\asz{Yes, definitely -- I'll take a look at that.}

\begin{definition}[SFI $\Cpc$ label]\label{def:sfi-pc-label}
  An \term{SFI $\Cpc$ label} is a tuple $(\Cpcstate, \Ccompartment)$, where
  $\Cpcstate$ is either
  \begin{enumerate}
  \item An indication of intra-compartment execution, $\concrete{INTERNAL}$; or
  \item An indication that the execution came from a (potentially
    inter-compartmental) $\opcode{jump}$, written $\concrete{JUMPED}$.
  \end{enumerate}
\end{definition}

Registers are always tagged with a dummy $\concrete{REG}$ tag, since they don't
need to carry any compartmentalization information.
%
\ch{This is quite sad. I think it would be more helpful to tag
  registers with the component that did the last write, and not allow
  inter-component reads; or at least require explicit retagging for
  this. This way the caller would no longer have to manually clear a
  ton of registers, while still allowing data to be passed via
  registers if necessary. Retagging could use instruction modifiers to
  achieve lower overhead. Anyway, maybe this is just a compromise for
  containing complexity.}%
\asz{Exposing manual register retagging at the abstract level gets messy, since
  right now the abstract machine has no notion of tags or of compartment IDs.
  I'm not sure what you mean by ``instruction modifiers''.}

We can use this to define the concrete SFI machine as the tagged concrete
machine we are expecting.

\begin{definition}[SFI concrete machine state]
\label{def:sfi-concrete-machine-state}
  An \term{SFI concrete machine state} $\CMachine$ is a triple
  $(\Cpc@\Label_{\Cpc},\CRegisters,\CMemory)$ of
  \begin{enumerate}
  \item A \term{program counter} $\Cpc \in \Z_{2^{32}}$ pointing at the
    to-be-executed instruction, with a tag $\Label_{\Cpc}$ indicating where the
    transition to this instruction came from;
  \item A finite set of \term{registers} $\CRegisters$ mapping registers to
    labeled values $\concrete{x}@\Label$; and
  \item A \term{memory} $\Memory$ mapping concrete addresses (elements of
    $\Z_{2^{32}}$) to labeled values $\concrete{x}@\Label$.
  \end{enumerate}
  We discuss invariants on the labels below.
\end{definition}

This is sufficient to emulate the above machine; the only difference is that
incorrectly changing the executing compartment (either by falling through to
code out of the current compartment or executing a disallowed $\opcode{jump}$)
fails one step later in this model.  Where in the abstract machine, we disallow
\emph{transitions} that would change the current compartment illegally, on this
machine we can only detect this when the $\Cpc$ tag disagrees with the tag on
the instruction it is about to execute.  Writes can be caught immediately,
however: while the TMU does not provide us the tag on the next instruction to be
executed, it does provide us the tag on the the memory location to be written.

To capture the essence of the can-execute check $\Compartments \vdash \pc,\pc+1
\in (\AddressSpace,\JumpTargets,\SharedMemory)$ we see in nearly all abstract
SFI machine instructions, we introduce the \term{can-execute} check seen in
\cref{fig:can-execute}; $\Label_{\Cpc} \canexec \Label_{\concrete{I}} \in
\Ccompartment$ precisely when
\begin{enumerate}
\item $\Label_{\Cpc}$ and $\Label_{\concrete{I}}$ are in the same compartment;
  or
\item $\Label_{\Cpc}$ indicates that we just jumped to the current instruction,
  and $\Label_{\concrete{I}}$ contains the $\Cpc$'s compartment as a valid
  incoming compartment.
\end{enumerate}

\begin{figure}
  \ottdefncanXXexecute{}
  \vspace{-\baselineskip}
  \caption{The \term{can-execute} check: does the current $\Cpc$ permit an
    instruction with the given data label to execute?  If so, in what
    compartment?}
  \label{fig:can-execute}
\end{figure}

The instructions that were simple in the abstract machine remain simple, as we
can see in \cref{fig:sfi-concrete-step-simple}.  One thing to note is that the
$\Cpc$ consistently changes labels from $\Label_{\Cpc}$ to
$(\INTERNAL,\Ccompartment)$.  In most cases, these two will be the same (making
this change a no-op), as the previous instruction will not have been a jump.
However, recall from \cref{def:sfi-pc-label} that jumping produces a special
$\JUMPED$ tag; fixing the return tags to be $\INTERNAL$ means that we do not
maintain the $\JUMPED$ tag past one instruction, as is appropriate.

\begin{figure}
  \begin{ottdefnblock}{$\mathcal{M}_{\concrete{C}} \ottsym{→} \mathcal{M}_{\concrete{C}}'$}
                      {\ottcom{$\mathcal{M}_{\concrete{C}}$ steps to $\mathcal{M}_{\concrete{C}}'$}}
    \ottusedrule{\ottdrulecstepXXnop{}}\medskip
    \ottusedrule{\ottdrulecstepXXconst{}}\medskip
    \ottusedrule{\ottdrulecstepXXmov{}}\medskip
    \ottusedrule{\ottdrulecstepXXbinop{}}\medskip
    \ottusedrule{\ottdrulecstepXXbnzXXzero{}}\medskip
    \ottusedrule{\ottdrulecstepXXbnzXXnonzero{}}
  \end{ottdefnblock}\vspace{-\baselineskip}
  \caption{The standard portion of the concrete SFI machine semantics.}
  \label{fig:sfi-concrete-step-simple}
\end{figure}

Looking at \cref{fig:sfi-concrete-step-jump}, we immediately see a difference
between the concrete and abstract machines: jumps to out-of-compartment
locations don't fail for that reason!  Indeed, as mentioned above, we cannot
inspect the tag on the next $\Cpc$ to find out if we have left our current
compartment.  Thus, we must compare the $\Cpc$ tag and the instruction tag,
which is what the can-execute relation does.  But to permit inter-compartment
movement, we must have $\opcode{jump}$ produces the $\JUMPED$ tag; this allows
the can-execute relation to notice that the previous instruction was a jump, and
permit it if necessary.

\begin{figure}
  \begin{ottdefnblock}{$\mathcal{M}_{\concrete{C}} \ottsym{→} \mathcal{M}_{\concrete{C}}'$}
                      {\ottcom{$\mathcal{M}_{\concrete{C}}$ steps to $\mathcal{M}_{\concrete{C}}'$}}
    \ottusedrule{\ottdrulecstepXXjump{}}
  \end{ottdefnblock}\vspace{-\baselineskip}
  \caption{Jumping on the concrete SFI machine.}
  \label{fig:sfi-concrete-step-jump}
\end{figure}

Accessing memory, as we can see in \cref{fig:sfi-concrete-step-memory}, is much
more similar to the abstract machine.  Loads and stores continue to be
\emph{asymmetric}, and $\opcode{store}$ \emph{can} check directly that the write
it's performing is permitted.

\begin{figure}
  \begin{ottdefnblock}{$\mathcal{M}_{\concrete{C}} \ottsym{→} \mathcal{M}_{\concrete{C}}'$}
                      {\ottcom{$\mathcal{M}_{\concrete{C}}$ steps to $\mathcal{M}_{\concrete{C}}'$}}
    \ottusedrule{\ottdrulecstepXXload{}}\medskip
    \ottusedrule{\ottdrulecstepXXstore{}}
  \end{ottdefnblock}\vspace{-\baselineskip}
  \caption{Memory access on the concrete SFI machine.}
  \label{fig:sfi-concrete-step-memory}
\end{figure}

Finally, the rule for $\opcode{isolate}$ is presented in
\cref{fig:sfi-concrete-step-isolate}.  The meaning of the arguments, and their
correctness conditions, are the same between the concrete and abstract machines;
however, because the tagging scheme used for the concrete machine differs from
the global state scheme used for the abstract machine, we go over the retagging
effects of this instruction.
\begin{enumerate}
\item We allocate a fresh compartment ID $\Ccompartment'$.
  
\item We retag the address space of the new compartment to lie in this new
  compartment.

\item The valid jump targets for the new compartment -- wherever they are in
  memory -- are all retagged to add it to their set of permissible incoming
  compartments.

\item Similarly, the shared memory locations available to the new compartment
  are all retagged to add it to their set of permissible writers.

\item Finally, the first instruction of the new compartment has the old
  compartment added as a permitted incoming compartment.
\end{enumerate}

\begin{figure}
  \begin{ottdefnblock}{$\mathcal{M}_{\concrete{C}} \ottsym{→} \mathcal{M}_{\concrete{C}}'$}
                      {\ottcom{$\mathcal{M}_{\concrete{C}}$ steps to $\mathcal{M}_{\concrete{C}}'$}}
    \ottusedrule{\hspace*{-2em}\ottdrulecstepXXisolate{}}
  \end{ottdefnblock}\vspace{-\baselineskip}
  \caption{Creating a new compartment on the concrete SFI machine.\asz{Wow, this
      rule needs to be cleaned up!}}
  \label{fig:sfi-concrete-step-isolate}
\end{figure}

\section{Invariants for the concrete machine}

\begin{definition}[SFI label invariants]\label{def:sfi-label-invariants}
  For a concrete SFI machine, we wish to maintain the following label
  invariants:
  
  \begin{enumerate}
  \item All registers have the dummy tag $\REG$.
  \item The $\Cpc$ has a $\Cpc$ tag (\cref{def:sfi-pc-label}), of the form
    $(\Cpcstate,\Ccompartment)$.
  \item All memory locations have a data tag (\cref{def:sfi-data-label}), of the
    form $(\Ccompartment,\CIncoming,\CWriters)$.
  \end{enumerate}
\end{definition}

\begin{lemma}[SFI register labels]\label{lem:sfi-register-labels}
  Given two concrete SFI machine states $\CMachine$ and $\CMachine'$ such that
  all of $\CMachine$'s registers are tagged $\REG$ and $\CMachine \stepsto
  \CMachine'$, so are all of $\CMachine'$'s.
\end{lemma}

\begin{lemma}[The SFI $\Cpc$ label]\label{lem:sfi-pc-label}
  Given two concrete SFI machine states $\CMachine$ and $\CMachine'$ such that
  all of $\CMachine$'s registers have tags of the form
  $(\Cpcstate,\Ccompartment)$ and $\CMachine \stepsto \CMachine'$, so do all of
  $\CMachine'$'s.
\end{lemma}

\begin{lemma}[SFI memory labels]\label{lem:sfi-memory-labels}
  Given two concrete SFI machine states $\CMachine$ and $\CMachine'$ such that
  all of $\CMachine$'s registers have tags of the form
  $(\Ccompartment,\CIncoming,\CWriters)$ and $\CMachine \stepsto \CMachine'$, so
  do all of $\CMachine'$'s.
\end{lemma}

\begin{theorem}[SFI label invariant maintained]
\label{thm:sfi-label-invariant-maintained}
  Suppose that we have two concrete SFI machine states $\CMachine$ and
  $\CMachine'$ such that $\CMachine$ respects the SFI label invariant
  (\cref{def:sfi-label-invariants}) and $\CMachine \stepsto^* \CMachine'$.  The
  $\CMachine'$ respects the label invariant as well.
\end{theorem}
\end{document}

% LocalWords: LocalWords SFI retag retagging retagged
